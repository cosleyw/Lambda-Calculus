\section{Motivation}

Why study lambda calculus?

Let us put the question in some historical context. At the beginning of the
20th century the famous mathematician David Hilbert challenged the
mathematical community by the statement that mathematical problems must be
decidable. At the 1930 annual meeting of the \emph{Society of German
  Scientists and Physicians} he made his famous quote ``We must know, we will
know''.

\noindent
\begin{tikzpicture}[scale=0.70]
  \def\pict#1#2#3#4{
    \node[fill=gray!20,text width=3.5cm,text centered] (#1) at #2
    {\scriptsize \includegraphics[width=3cm,height=4cm]{#3} #4};
  }
  \pict{hilbert}{(0,7)}{hilbert.jpg}{David Hilbert}

  \node[draw,fill=gray!30,text width=6cm] (text) at (10,8) {``Entscheidungsproblem''
    (Decision Problem). Mathematics must be decidable. ``We must know, we will
    know!''};

  \pict{goedel}{(0,0)}{goedel.jpg}{Kurt Gödel~(1931): Incompleteness Theorems}
  \pict{church}{(6,0)}{church.jpg}{Alonzo~Church~(1936): Lambda Calculus}
  \pict{turing}{(12,0)}{turing.jpg}{Alan Turing~(1936): Turing Machine}

  \draw [thick,->] (hilbert) -- (text) node[above,midway,sloped]{Challenge};
\end{tikzpicture}


The young mathematician Kurt Gödel attended the meeting and expressed some
doubts to his collegues about the general decidability of mathematical
statements. One year later in 1931 he published his famous incompleteness
theorems~\cite{goedel1931}. He proved that for all consistent formal systems
which are capable of expressing logic and doing simple arithmetics there are
certain statements which are not provable withing the system but true. These
incompleteness theorems are considered as the first serious blow of Hilbert's
program.

Five years later Alonzo Church~\cite{church1936} and Alan
Turing~\cite{turing1936} independently proved that the \emph{decision problem}
cannot be solved. Alonzo Church invented the lambda calculus and Alan Turing
his automatic machine (today called Turing machine) which are both equivalent
in expressiveness.

Although Church's lambda calculus has been published slightly before Alan
Turing published his paper on automatic machines usually Turing machines are
used define computability and decidability. Turing machines resemble more the
structure of modern computers than lambda calculus. A programming language is
called \emph{Turing complete} if all possible algorithms can be coded within
the language. Nobody talks about \emph{lambda complete}.

However lambda calculus is a quite fascinating model of computation.  The
lambda calculus invented by Alonzo Church is remarkably simple. It consists
just of variables, function applications and lambda abstractions. But the
calculus is sufficiently powerful to express all computable functions and
decision procedures.

Beside its expressive power lambda calculus is used as the theoretical base of
functional languages like Haskell, ML, F\#.

In this paper we explain the lambda calculus in its purest form as untyped
lambda calculus.

%%% Local Variables:
%%% mode: latex
%%% TeX-master: "main_untyped_lambda"
%%% End:
