\section{Lambda Terms}
\label{sec:lambda}

\subsection{Basic Definitions}

Imagine a mathematical function with one argument which triplicates the
argument and adds five to the result. How would you write such a function. In
mathematics the most straightforward notation is
$$ x \mapsto 3 \times x + 5.$$
%
The name of the variable $x$ is not important. We could write the same
function as $y \mapsto 3\times y + 5$. The variables $x$ and $y$ are called bound
variables because they are bound by the context definining the function.

Now suppose you want to apply the function to an actual argument, say
$2$. I.e. we want to compute $(x \mapsto 3\times x + 5)(2)$. We do it by
replacing the variable $x$ in the expression $3\times x + 5$ defining the
function by the argument $2$ resulting if $3\times 2 + 5$.

More formally we could write
$$ (x \mapsto 3 \times x + 5)(2) \to (3 \times x + 5)[x:=2] = 3\times 2 + 5.$$
where $\to$ means \emph{reduces to}.

That is already the essence of lambda calculus. In lambda calculus we write
the function $$x \mapsto 3 \times x + 5$$ as $$\lambda x. 3 \times x + 5.$$
and we write function application by juxtaposition
$$ (\lambda x. 3\times x + 5) 2$$
which reduces to
$$ (\lambda x. 3\times x + 5) 2 \to_\beta (3 \times x + 5)[x:=2]$$.
The application of the function to an argument is called a $\beta$-reduction.

$\beta$-reduction is done by variable substitution which can be done purely
mechanically i.e. it is the essence of a computation step.

In lambda calculus we have no primitive data types like booleans, numbers
pairs etc. There are only functions. However it is possible to represent data
by functions as we shall see later. Data are represented by functions which
capture the essence of what can be done by the data.

E.g. boolean values can be used to decide between two alternatives. Therefore
a boolean value is represented in lambda calculus by a function with two
arguments which chooses the first or second argument depending on its value.

Numbers are represented by functions which take two arguments, a function and
a start value and the lambda term representing the number iterates the
function n-times on the start value.



\paragraph{Definition of Lambda Terms}

\begin{definition}
  Let $x$ range over a countably infinite set of variable names
  $\{x_0, x_1, \ldots\}$ and $t$ over lambda terms, then the set of lambda
  terms is defined by the grammar $$t ::= x \mid t\, t \mid \lambda x. t.$$
\end{definition}

A lambda term is either a variable $x$, an application $a\,b$ (the term $a$
applied to the term $b$) or an abstraction $\lambda x.a$.

We use the convention that application is left associative i.e. $a b c$ is
parsed as $(a b) c$.

Nested lambda abstractions $\lambda x. \lambda y. \ldots . t$
are parsed as $\lambda x. (\lambda y. \ldots . t)$ and
abbreviated as $\lambda x y \ldots\, . t$.


\paragraph{Free and Bound Variables}

In the abstraction $$ \lambda x. t$$ the variable $x$ is a bound variable. It
is not vlsible to the outside world. This is the same convention as used in
programming languages which allow the definition of procedures. The formal
arguments names of the procedure/function arguments are just visible to the
definition of the procedure/function and not to the outside world.

Variables which are not bound by a lambda abstraction are free variables.

\begin{definition}
  The set of free variables $FV(t)$ of a lambda term $t$ is defined by
  $$FV(t) :=
  \begin{cases} FV(x) &= \{x\} \\
     FV(a b) &= FV(a) \cup FV(b) \\
     FV(\lambda x. t) &= FV(t) - \{x\}
   \end{cases}
   $$
\end{definition}

\begin{definition}
  A lambda term without free variables is called a \emph{closed lambda term}.
\end{definition}

Evidently bound variables can be renamed without changing the meaning of the
term. E.g. the two lamda terms
$$\lambda x. x $$
$$\lambda y. y  $$
are considered as the same term which represents the identity
function. Traditionally the terms are called $\alpha$-equivalent because you
transform one into the other by just renaming bound variables.

We write $t = u$ only if $u$ and $t$ are exactly the same term or
$\alpha$-equivalent terms.

Renaming of bound variables must be done in a way which does not change the
structure of the term. The following two rules must be obeyed.
\begin{enumerate}

\item Keep different bound variables distinct.
  $$\begin{array}{llll}
      \text{legal:} & \lambda x y. x\, y  &  \text{rename to}  & \lambda a
                                                                 b. a\, b \\
      \text{illegal:} & \lambda x y. y      &  \text{rename to}  & \lambda x
                                                                   x. x
    \end{array} $$

\item Do not capture free variables.
  $$\begin{array}{llll}
      \text{legal:} & \lambda x. x\, y  &  \text{rename to}  & \lambda z. z\,y
                                                                \\
      \text{illegal:} & \lambda x. x\,y &  \text{rename to}  & \lambda y. y\,y
    \end{array} $$
    The second rename renames the variable $x$ into the variable $y$ which as
    originally a free variable but captured after the rename.
\end{enumerate}


\paragraph{Variable Substitution}

\begin{definition}
  The variable substitution $a[x:=t]$ is defined by
  $$a[x:=t]~:=
  \begin{cases} x[x:=t]  &:= t \\
    y[x:=t] &:= y \quad \text{for}\quad x \ne y \\
    (a b)[x:=t] &:= a[x:=t] \, b[x:=t] \\
    (\lambda y.a)[x:=t]  &:= \lambda y. a[x:=t] \quad\text{for}\quad x \ne y
    \land y \notin FV(t)
   \end{cases}
   $$
\end{definition}

Note: The condition on the last line is no restriction because we can always
rename the bound variable $y$ to a fresh variable $z$ different from $x$ and
not occuring free in $t$ since there are infinitely many variables available.


\paragraph{Substitution Swap Lemma}

The expression
$$a[x:=b][y:=c]$$ describes the term $a$ where in a first step the variable
$x$ is substituted by the term $b$ and then in a second step the variable $y$
is substituted by the term $c$. Usually it is assumed that $x$ and $y$ are
different variables and that $x$ does not occur free in $c$, i.e. $x\ne y
\land x \notin FV(c)$.

However two subsequent substitutions do not commute. The term
$$ a[y:=c][x:=b]$$ is in general different from the previous term.
Reason: Neither $a[x:=b][y:=c]$ nor $a[y:=c]$ do contain any
$y$. But $b$ might contain $y$ and therefore $a[y:=c][x:=b]$ might contain
$y$. In order to make the swapping correct we have to do the substitution
$b[y:=c]$ before substituting the variable $x$ by $b$.

\begin{theorem}
  \label{substitutionswap}
  Substitution Swap lemma: Let $x \ne y$ and $x \notin
  FV(c)$. Then $$a[x:=b][y:=c] = a[y:=c]\big[x:= b[y:=c]\big]$$. Proof by
  induction on the structure of $a$. We use the abbreviations
  $$\begin{array}{ll}
      s_1(a) &:= a[x:=b][y:=c] \\
      s_2(a) &:= a[y:=c]\big[x:=b[y:=c]\big]
    \end{array}.$$
  \begin{enumerate}

  \item $a$ is a variable. Lets call it $z$. Goal $s_1(z) = s_2(z)$
    \begin{itemize}
    \item $z \ne x \land z \ne y$: $s_1(z) = z = s_2(z)$
    \item $z = x \land z \ne y$: $s_1(z) = b[y:=c] = s_2(z)$
    \item $z \ne x \land z = y$: $s_1(z) = c = s_2(z)$
    \end{itemize}

  \item $a$ is the application $t\, u$. Goal $s_1(t\,u) = s_2(t\,u)$.
    Induction hypotheses $s_1(t) = s_2(t)$ and $s_1(u) = s_2(u)$
    $$\begin{array}{llll}
        s_1(t\,u) &=& s_1(t) s_1(u) & \text{definition of substitution}\\
       &=& s_2(t) s_2(u) & \text{induction hypothesis}\\
        &=& s_2(t\,u) & \text{definition of substitution}
      \end{array}$$

    \item $a$ is the abstraction $\lambda z.t$. Goal $s_1(\lambda z.t) =
      s_2(\lambda z.t)$. Induction hypothesis $s_1(t) = s_2(t)$.
    $$\begin{array}{llll}
        s_1(\lambda z. t) &=& \lambda z. s_1(t)  & \text{definition of substitution}\\
       &=& \lambda z. s_2(t) & \text{induction hypothesis}\\
        &=& s_2(\lambda z.t) & \text{definition of substitution}
      \end{array}$$
      with appropriate renaming of the bound variable $z$ in order to avoid
      variable capture (i.e. $z$ must be different from $x$ and $y$ and must
      not occur free neither in $a$ nor in $b$).
  \end{enumerate}
\end{theorem}




\paragraph{Beta Reduction}

Now we are able to define the essential computation step in lambda calculus
which is beta reduction. Any term of the form
$$ (\lambda x.a) b$$ is called a reducible expression or in short a
\emph{redex} which reduces in one step to
$$a[x:=b].$$ The redex can appear anywhere inside a lambda term.

\begin{definition} \emph{Beta reduction} $\tobeta$ is a relation defined over lambda
  terms by the rules
  \begin{enumerate}
  \item $(\lambda x.a) b \tobeta a[x := b]$
  \item $\rulev{a\tobeta b}{a c \tobeta b c}$
  \item $\rulev{b\tobeta c}{a b \tobeta a c}$
  \item $\rulev{a \tobeta b}{\lambda x.a \tobeta \lambda x.b}$
  \end{enumerate}
\end{definition}

Beta reduction $\to_\beta$ is a one step relation. The expression
$t \to_\beta^+ u $ states that $t$ can be reduced to $u$ in one or more
$\beta$-reduction steps.  The expression $t \to_\beta^* u $ states that $t$
can be reduced to $u$ in zero or more $\beta$-reduction steps.

Two terms $t$ and $u$ are called $\beta$-equivalent if $t \to_\beta^\sim u$ is
valid. Recall from section~\ref{setrelation} that the equivalence
closure~\ref{def:equivalenceclosure} means that $t$ can be transformed into
$u$ by using zero or more beta reduction steps in forward (reduction) or
backward (expansion) direction.

\paragraph{Normal Forms}

\begin{definition}
  A $\lambda$-term is in \emph{normal form} if it is a terminal element of the
  $\beta$-reduction relation.

  A $\lambda$-term is \emph{normalizing} if it is a weakly terminating element of the
  $\beta$-reduction relation.

  A $\lambda$-term is \emph{strongly normalizing} if it is a strongly
  terminating element of the $\beta$-reduction relation.
\end{definition}
In other words
\begin{itemize}
\item A $\lambda$-term is in normal form if it contains no reducible expression.
\item A $\lambda$-term $t$ is normalizing if there is a reduction path of zero
  or more steps $t \to_\beta^* u$ where $u$ is in normal form.
\item A $\lambda$-term $t$ is strongly normalizing if there is all reduction
  paths end up after zero or more steps in some normal form.
\end{itemize}

Clearly all terms in normal form are trivially normalizing and strongly
normalizing.




\subsection{Simple Computation with Combinators}

In this subsection we demonstrate how lambda terms can be used to do simple
computations. We base our terms on combinators which are closed lambda terms
i.e. terms without free variables.

The simplest combinator is the identity combinator defined as

$$ I := \lambda x.x $$

where $I$ is just an abbreviation for the term on the right hand side of the
definition. The lambda calculus does not know the term $I$, it just knows
terms like $\lambda x.x$. We use $I$ for us to formulate the calculus more
readable for humans.

The identity function takes one argument and returns exactly the same
argument which can be proved by application of the rules for
$\beta$-reduction
$$\begin{array}{llll}
  I a & = &  (\lambda x.x) a & \text{definition of $I$} \\
       & \to_\beta & x[x:=a]  & \text{rule 1 of $\beta$-reduction} \\
       & =              & a           & \text{definition of substitution}
\end{array}.$$

The mockingbird combinator is defined as
$$ M := \lambda x. x x.$$
Birdnames are used in this text as the names for combinators to honor Haskell
Curry who is one of the inventors of combinatorial logic and who loved to
watch birds and to honor Raymond Smullyan who wrote the book \emph{To Mock a
  Mockingbird}~\cite{smullyan} using birds and forests and puzzles about them
to teach combinatorial logic in an entertaining and amusing way.

The mockingbird combinator receives one argument and applies it to itself. The
term $M M$ has the interesting property to reduce to itself
$$\begin{array}{llll}
  M M & = &  (\lambda x.x x) M  & \text{definition of $M$} \\
       & \to_\beta & (x x)[x:=M]   & \text{rule 1 of $\beta$-reduction} \\
       & =              & M M              & \text{definition of substitution}
\end{array}$$
so that we have
$$ MM \to_\beta MM \to_\beta MM \ldots$$
which represents the simplest form of an \emph{endless loop} in lambda
calculus.

A very important combinator is the kestrel
$$ K := \lambda x y. x$$
which receives two arguments and returns the first, easily proved by
$$\begin{array}{llll}
    K a b & = & (\lambda x y. x) a b & \text{definition of $K$}  \\
          & = &(\lambda x. \lambda y. x) a b & \text{shorthand expanded} \\
          & \to_\beta & (\lambda y.x)[x:=a]\,b & \text{rule 1 of
                                                   $\beta$-reduction}\\
          & = & (\lambda y. a) b & \text{definition of substitution} \\
          & \to_\beta & a[y:=b] & \text{rule 1 of $\beta$-reduction} \\
          & = & a & \text{definition of substitution}
  \end{array}.$$

The kestrel shows that $\lambda$-terms can in some way \emph{store} values. If
we apply the kestrel $K$ only to one argument $a$ we get $\lambda y.a$. This
term stores the value $a$ within the abstraction. If later the term
receives its second argument it spits out the stored value $a$ ignoring its
second argument.

The companion of the kestrel is the kite with the definition
$$ K_I := \lambda x y. y$$ which receives two arguments and returns always the
second i.e.
$$ K_I a b \to_\beta^+ b$$
which can be proved in a similar manner.

A combinator with the same behaviour as the kite can be constructed as an
application of the kestrel to the identity function
$$ K I$$
where
$$ K I \to_\beta \lambda y. I$$
so that we get
$$ K I a b \to_\beta (\lambda y. I) a b \to_\beta I b \to_\beta b.$$

Note that $K I$ is not $\alpha$-equivalent to $K_I$, but both terms are
$\beta$-equivalent because they reduce to the $\alpha$-equivalent terms
$\lambda y x.x$ and $\lambda x y. y$.

The kestrel applied to one argument stores the argument a returns it by ignoring the
second argument. The trush stores its first argument as well but in a more
interesting manner
$$ T := \lambda x f. f x.$$
The trush stores its first argument and waits until it receives its second
argument. After receiving its second argument it uses the second argument as a
function and applies it to the first argument.

Even more interesting in its storage behaviour is the vireo, defined as
$$ V := \lambda x y f. f x y.$$

The vireo applied to two arguments stores the arguments (i.e. it stores a pair
of values). After receiving its third argument it applies the third argument
as a function to the two stored values. Combining the vireo, the kestrel and
the kite we can encode pairs. $V a b$ stores the pair $(a,b)$ and $V a b K$
returns the first element of the pair and $V a b K_I$ returns the second
element of the pair i.e. $V a b K \to_\beta^+ a$ and $V a b K_I \to_\beta^+ b$
which can be proved by using the definitions and applying $\beta$-reduction
and substitution.

\paragraph{Summary of Combinators}

Computing with $\lambda$-terms is purely mechanical, but it can be tedious if
the terms become more complicated. Combinators serve as a kind of abstraction
layer to make it easier to manipulate and prove assertions about lambda terms.

Having a definition of an combinator e.g. the kestrel $$K := \lambda x y.x,$$
the concrete definition is usually not necessery once the crucial property of
the kestrel $$K a b \to_\beta^+ a$$ has been proved to be valid.

In this text we don't use complicated $\lambda$-terms. We always use
combinators with their corresponding properties to express compactly our
claims and proves of these claims.

In the following table the most important combinators together with their
specifications and implementations are summarized.

\begin{tabular}{|l|l|l|l|}
  \hline
  Name & Abbreviation & Specification & Implementation \\
  \hline\hline
  Bluebird & $B$ & $B f g x \to_\beta^+ f (g x)$   & $\lambda f g x. f (g x)$ \\
  \hline
  Identity & $I$ & $I x \to_\beta x$   & $\lambda x. x$ \\
  \hline
  Kestrel & $K$ & $K x y \to_\beta^+ x$   & $\lambda x y. x$ \\
  \hline
  Kite & $K_I$ & $K_I x y \to_\beta^+ y$   & $\lambda x y. y$ \\
  \hline
  Mockingbird & $M$ & $M x \to_\beta^+ x \, x$   & $\lambda x. x\,x$ \\
  \hline
  Starling & $S$ & $S f g x\to_\beta^+ f x (g x)$   & $\lambda f g x. f x(g x)$ \\
  \hline
  Trush & $T$ & $T x f\to_\beta^+ f\,x$   & $\lambda x f. f\,x$ \\
  \hline
  Turing  & $U$ & $U x f\to_\beta^+ f (x x f)$   & $\lambda x f. f (x x f)$ \\
  \hline
  Vireo & $V$ & $V x y f\to_\beta^+ f x y$   & $\lambda x y f. f x y$ \\
  \hline
\end{tabular}







\subsection{Confluence - Church Rosser Theorem}

A lambda term might contain more than one reducible expression. The
$\beta$-reduction relation is therefore non-deterministic, you can choose any
reducible expression to do a $\beta$-reduction step.

Therefore the question arises, if different reduction paths end up at the same
result. Is $\beta$-reduction confluent?

Remember that confluence is a rather strong property as explained in the
section \emph{Inductive Sets and Relations}~\ref{setrelation}. Confluence
guarantees the uniquess of terminal elements (or normal forms in
$\lambda$-calculus speak) provided that they exist.

In order to prove that $\beta$-reduction is confluent we have to prove that
$\to_\beta^*$ is a diamond, i.e. that
$$\begin{matrix}
  a & \to_\beta^* & b \\
  \downarrow_\beta^* & & \downarrow_\beta^* \\
  c & \to_\beta^* & \exists d
\end{matrix}$$
is valid.


\paragraph{$\beta$-reduction is not a diamond}

If $\to_\beta$ were a diamond we would be ready, because a diamond is
confluent as proved in~\ref{th:diamondconfluent}. Unfortunately $\to_\beta$ is
not a diamond for the following reason:

The core of the reduction relation is $(\lambda x.a) b \to_\beta a[x:=b]$
where $(\lambda x.a) b$ is the reducible expression. Both subterms $a$ and $b$
might contain further reducible expressions. Since the variable $x$ might be
contained in the expression $a$ zero, one or more times all reducible
expressions of $b$ can be contained in $a[x:=b]$ zero, one or more times.

If $b$ contains a reducible expression a reduction step $b \to_\beta c$ is
possible for some $c$. We have the situation that two reduction paths are
possible
$$
\begin{matrix}
  (\lambda x.a) b & \to_\beta & a[x:=b] \\
  \downarrow_\beta & & \\
  (\lambda x.a) c  & \to_\beta & a[x:=c]
\end{matrix}.$$

There are 3 cases:
\begin{itemize}
\item
  Case variable $x$ does not occur in $a$: Then $a[x:=b] $ and $a[x:=c]$
  are the same expression and since $\to_\beta$ is not reflexive there is no
  way to complete the diagram.

\item
  Case variable $x$ does occur once in $a$: Then $a[x:=b] \to_\beta
  a[x:=c]$ is a valid reduction step and the diagram can be completed.

\item
  Case variable $x$ does occur 2 or more times in $a$: Then $a[x:=b]$ cannot
  be reduced to $a[x:=c]$ in one step. 2 or more steps are necessary.
\end{itemize}



\paragraph{Properties of $\to_\beta^*$}
As explained in the section \emph{Diamonds and
  Confluence}~\ref{sec:diamondsconfluence}
we can search for properties of  $\to_\beta^*$ which indicate that
$\to_\beta^*$ is a diamond and use these properties to construct a diamond
between $\to_\beta$ and $\to_\beta^*$.

$\to_\beta^*$ has the property that it can do zero or more reduction steps in
parallel in \emph{any} subexpression of a lambda expression. I.e. intuitively
the following rules are valid:
\begin{enumerate}
\item
  $a \to_\beta^* a$
\item
  $\ruleh {a \to_\beta^* b} {\lambda x.a \to_\beta^* \lambda x.b}$
\item
  $\ruleh {a \to_\beta^* b, \, c \to_\beta^* d} {ac \to_\beta^* bd}$
\item
  $\ruleh {a \to_\beta^* b, \, c \to_\beta^* d}  {(\lambda x.a) c \to_\beta^* b[x:=d]} $
\end{enumerate}

Although evident by intuition we have to prove these properties.

\begin{theorem}
  $\tobetastar$ satisfies
  $\ruleh{
    a\tobetastar b \quad
    c \tobetastar d}
  {ac \tobetastar bd}$.
  %---------------

Proof by induction on $a \tobetastar b$.
\begin{enumerate}
\item
  Goal $ac \tobetastar ad$ assuming $c  \tobetastar d$.
  Proof by subinduction on $c \tobetastar d$
  \begin{enumerate}
  \item
    Case $c = d$. Trivial by reflexivity.
  \item
    Goal $ac \tobetastar ae$.
    Premises $c \tobetastar d$ and $d \tobeta e$.
    Induction hypothesis $ac \tobetastar ad$.\\
    $\begin{bmatrix}
      c   & \tobetastar      &  d     & \tobeta            &   e   \\
           & \Downarrow_1 &         & \Downarrow_2 &        \\
      ac & \tobetastar      &  ad   & \tobeta             &   ae \\
          &                         & \Downarrow_3 &          &        \\
      ac &                        & \tobetastar   &              &   ae
    \end{bmatrix}$\\
    $\Downarrow_1$ by induction hypothesis.
    $\Downarrow_2$ by rule 3 of $\beta$-reduction.
    $\Downarrow_3$ by rule 2 of reflexive transitive closures.
  \end{enumerate}
\item
  Goal $ac \tobetastar ed$.
  Premises $a \tobetastar b$ and $b\tobeta e$.
  Induction hypothesis $ac \tobetastar bd$.\\
    $\begin{bmatrix}
      a   & \tobetastar      &  b     & \tobeta            &   e   \\
           & \Downarrow_1 &         & \Downarrow_2 &        \\
      ac & \tobetastar      & bd    & \tobeta            &   ae \\
           &                        & \Downarrow_3 &         &        \\
      ac &                         & \tobetastar   &           &   ae
    \end{bmatrix}$.\\
    $\Downarrow_1$ by induction hypothesis.
    $\Downarrow_2$ by rule 2 of $\beta$-reduction.
    $\Downarrow_3$ by rule 2 of reflexive transitive closures.
  \end{enumerate}
\end{theorem}

\begin{theorem}
  $\tobetastar$ satisfies
  $\ruleh{
    a\tobetastar b \quad
    c \tobetastar d}
  {(\lambda x.a)c \tobetastar b[x:=d]}$.
  %-----------------------------
  \\ Proof in the same manner as the previous theorem with induction on
  $a\tobetastar b$ and then a subinduction on $c \tobetastar d$ for the
  reflexive case.
\end{theorem}


\begin{theorem}
  $\tobetastar$ satisfies
  $\ruleh{
    a\tobetastar b}
  {\lambda x.a \tobetastar \lambda x.b}$.\\
  %-----------------------------
Proof in the same manner as the previous theorems without the need of a
subinduction because there is only one premise.
\end{theorem}


\paragraph{Definition of Parallel $\beta$-reduction}

Now that we have the found the properties of $\tobetastar$ which point into
the direction that it is a diamond we can use these properties as rules to
define the least relation satisfying these properties.


\begin{definition}
\label{def:parallelbeta}
  % -------------------------------------------------------
  \emph{Parallel beta reduction} $\tobetap$ is a relation
  defined over lambda terms by the rules
  % -------------------------------------------------------
  \begin{enumerate}
  \item $a \tobetap a$
  \item $\rulev{a \tobetap b} {\lambda x.a \tobetap \lambda x.b}$
  \item $\rulev{a\tobetap c \\ b \tobetap d}{a b \tobeta c d}$
  \item $\rulev{a\tobetap c \\ b \tobetap d}{(\lambda x.a) b \tobetap c[x := d]}$
  \end{enumerate}
\end{definition}

Obviously $\beta$-reduction is a subset of parallel
$\beta$-reduction.
\begin{lemma}
  Beta reduction is a subset of parallel beta reduction i.e.
  %---------------------------------------
  $a \tobeta b \imp a \tobetap b$.
  Proof by induction on $a \tobeta b$. Trivial because each rule of $\tobeta$
  is a special case of some rule of $\tobetap$.
\end{lemma}



\paragraph{Parallel $\beta$-Reduction is a Diamond}

In order to prove that $\tobetap$ is a diamond we need some lemmas.

\begin{lemma}
  Parallel beta reduction preserves abstraction i.e.
  $\lambda x.a \tobetap c \imp \exists b : a \tobetap b \land c = \lambda
  x.b$. Proof by induction on $\tobetap$.
  \begin{enumerate}
  \item $c = \lambda x.a$. Trivial. Take $b = a$.
  \item $\lambda x.a \tobetap \lambda x.b$ with $a \tobetap b$. Trivial. Take $b$.
  \item The case $\lambda x.a = t u$ is syntactically impossible. Abstraction
    and application are different.
  \item The case $\lambda x.a = (\lambda x.u) v$ is syntactically
    impossible. Abstraction and application are different.
  \end{enumerate}
\end{lemma}


\begin{lemma}
  Basic compatibility of substitution and parallel reduction.
  $t \tobetap u \imp a[x := t] \tobetap a[x := u]$. Proof by induction on the
  structure of $a$.
  \begin{enumerate}
  \item $a$ is a variable. Goal $z[x := t] \tobetap z[x := u]$. Case $z=x$
    is satisfied because of the assumption $t \tobetap u$. Case $z\ne x$ is
    satisfied by reflexivity $z \tobetap z$.
  \item
    $a$ is an application $b\, c$.
    Goal $(b c)[x:=t] \tobetap (b c)[x:=u]$
    $$
    \begin{array}{llll}
      (b c)[x:=t] &=& b[x:=t]\, c[x:=t] &              \text{definition of substitution}\\
                      &\tobetap & b[x:=u]\, c[x:=u] & \text{ind hypo + rule 3} \\
                      &=& (b c)[x:=u] &                       \text{definition of substitution}
    \end{array}
    $$
  \item
    $a$ is an abstraction $\lambda y.b$.
    Goal $(\lambda y.b)[x:=t] \tobetap (\lambda y.b)[x:=u]$.
    $$
    \begin{array}{llll}
      (\lambda y.b)[x:=t] &=& \lambda y. b[x:=t]  &\text{definition of substitution}\\
      &\tobetap & \lambda y.b[x:=u] &\text{ind hypo + rule 2} \\
      &=& (\lambda y.b)[x:=u] & \text{definition of substitution}
    \end{array}
    $$
  \end{enumerate}
\end{lemma}


\begin{lemma}
  Full compatibility of substitution and parallel reduction.
  % a -> c and b -> d  =>  a[x:=b] -> c[x:=d]
  % ----------------------------------
  $a \tobetap c \land b \tobetap d  \imp a[x := b] \tobetap c[x := d]$. Proof
  by induction on $a \tobetap c $.
  \begin{enumerate}
  \item
    $a=c$. Prove by the previous lemma.
  \item
    Goal $(\lambda y.a) [x := b] \tobetap (\lambda y.c) [x := d]$.
    Premise $a \tobetap c$. Induction
    hypothesis $a[x := b] \tobetap c[x := d]$
    $$
    \begin{array}{llll}
      (\lambda y.a)[x := b]  &=& \lambda y.a [x := b] &\text{definition substitution} \\
                             &\tobetap& \lambda y.c[x := d]  &\text{induction hypo + rule 2} \\
                             &=& (\lambda y.c)[x := d]           &\text{definition substitution}
    \end{array}
    $$.
  \item
    Goal $(a e) [x := b] \tobetap (c f) [x := d]$.
    Premises $a\tobetap c, e\tobetap f$.
    Induction hypotheses
      $a[x := b] \tobetap c[x := d],
       e[x := b] \tobetap f[x := d]$
    $$
    \begin{array}{llll}
      (a e)[x := b]  &=& a[x := b]\, e[x := b] &\text{definition substitution}\\
                     &\tobetap& c[x := d] \, f[x := d] &\text{induction hypo + rule 3} \\
                     &=& (c f)[x := d] &\text{definition substitution}
    \end{array}
    $$.
  \item
    Goal $\big((\lambda y.a) c\big) [x := b] \tobetap \big(e[y:=f]\big)[x := d]$.
    Premises $a \tobetap e, c \tobetap f$.
    Induction hypotheses
      $a[x:=b] \tobetap e[x := d],
        c[x:=b]  \tobetap f[x := d]$
    $$
    \begin{array}{llll}
      ((\lambda y.a) c)[x := b]  &=& (\lambda y.a[x := b])\, c[x := b] &
                                                                         \text{definition substitution}\\
                     &\tobetap& e[x := d]\big[y:=f[x := d]\big] &
                                                                  \text{induction hypo + rule 4} \\
                     &=& \big(e[y:=f]\big)[x := d] &
                                                                  \text{substitution
                                                           swap lemma}
    \end{array}
    $$
  \end{enumerate}
  $\qed$
\end{lemma}





\begin{theorem}
  Parallel reduction $\tobetap$ is a diamond i.e.
  %------------------------------------
  $\begin{matrix}
    a & \tobetap & b \\
    \downarrow_p & & \downarrow_p \\
    c & \tobetap & \exists d
  \end{matrix}
  $.
  Proof by induction on $a \tobetap b$.
  \begin{enumerate}
  \item
    Trivial reflexive case.
    $\diamp
    {a} {a}
    {c} {c}
    $

  \item Goal
    $\diamp
    {\lambda x.a} {\lambda x.b}
    {\lambda x.c} {?}$.
    Parallel reduction preserves abstraction. Therefore the specific
    $\lambda x.c$ instead of the more general $c$.
    The premises $a\tobetap b, a \tobetap c$ and the induction hypothesis
    guarantee the existence of a $d$ such that $\lambda x.d$ is the element
    to fill the gap.

  \item
    Goal $\diamp {a e} {b f} {c}  {?}$. Premises $a \tobetap b, e \tobetap f$.
    Proof by subinduction on $a e \tobetap c$.

    \begin{enumerate}
    \item Trivial reflexive case.
    \item Syntactically impossible
    \item
      Goal $\diamp {a e} {b f} {g h} {?}$.

      The premises $a \tobetap b, a \tobetap g, e \tobetap f, e \tobetap h$
      with the corresponding induction hypotheses guarantee the existence of
      two element $k$ and $m$ so that $ k m$ can fill the missing element in
      the goal.

    \item
      Goal $\diamp {(\lambda x.a) e} {(\lambda x.b) f} {c[x:=g]} {?}$.
      Parallel reduction preserves abstraction. Therefore the specific
      $\lambda x.b$ in the upper right corner.
      The premises $a \tobetap b, a \tobetap c, e \tobetap f, e \tobetap g$
      and the corresponding induction hypotheses guarantee the existence of
      two elements $d$ and $h$ so that $d[x:=h]$ fills the missing element
      in the goal.
    \end{enumerate}

  \item
    Goal $\diamp {(\lambda x.a) e} {b[x:=f]} {c} {?}$. Premises $a \tobetap b, e \tobetap f$.
    Proof by subinduction on $(\lambda x.a) e \tobetap c$.
    \begin{enumerate}
    \item Trivial reflexive case.
    \item Syntactically impossible
    \item Mirror image of case 3d, just flipped at the northwest-southeast diagonal.
    \item
      Goal $\diamp {(\lambda x.a) e} {b[x:=f]} {c[x:=g]} {?}$
      The premises $a \tobetap b, a \tobetap c, e \tobetap f, e \tobetap g$
      and the corresponding induction hypotheses guarantee the existence of
      two elements $d$ and $h$ so that $d[x:=h]$ fills the missing element
      in the goal.
    \end{enumerate}

  \end{enumerate}
  $\qed$
\end{theorem}

\paragraph{$\beta$-Reduction is Confluent}

\begin{theorem}
  Beta reduction is confluent.
  %--------------------
  Proof: With the parallel beta reduction$\tobetap$ we have found a diamond
  relation between beta reduction $\tobeta$ and its transitive closure
  $\tobetastar$. According to the confluence theorems of the chapter
  ``Inductive Sets and Relations'' this is sufficient to prove the confluence
  of beta reduction.
\end{theorem}

%%% Local Variables:
%%% mode: latex
%%% TeX-master: "main_untyped_lambda"
%%% End:
