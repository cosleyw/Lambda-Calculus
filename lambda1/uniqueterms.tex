\section{Unique Terms}

\begin{definition}
  Let $x$ range over a countably infinite set of variables
  $\{x_0, x_1, \ldots\}$ then the set of \emph{unique lambda terms} is defined by the
  grammar
  %
  $$t ::= x \mid t t \mid \lambdaw t$$
with the shorthands $\lambdaw^0 a = a$, $\lambdaw^{n'} a = \lambdaw \lambdaw^n a$.
\end{definition}


% Upshifting
% --------
\begin{definition} We define $a\upshift_i^n$ as the term $a$ where all
  variables above or equal $x_i$ are \emph{upshifted} by $n$ according to
$$ (\upshift) :=
        \begin{cases}
          x_k\upshift^n_i &:=
          \begin{cases}
            x_k, & k < i \\
            x_{k+n} & i \le k
          \end{cases}
          \\
          (a b)\upshift^n_i &:= a\upshift^n_i \, b\upshift^n_i
          \\
          (\lambdaw a) \upshift^n_i &:= \lambdaw a \upshift^n_{i'}
        \end{cases}
$$
and use the shorthands $t\upshift = t\upshift_0^1$
and
$t\upshift^n = t\upshift^n_0$.
\end{definition}




% Variable substitution
% -----------------
\begin{definition}
  The \emph{variable substitution} $a\lhd_i t$ is the term $a$ where the
  variable $x_j$ has been substituted by the term $t\upshift^i$ according to
$$ (\lhd) :=
        \begin{cases}
          x_k\lhd_i t &:=
          \begin{cases}
            x_k, & k < i
            \\
            t\upshift^i, & k = i
            \\
            x_{k-1} & i < k
          \end{cases}
          \\
          (a b)\lhd_i t &:= a\lhd_i t\, b\lhd_i t
          \\
          (\lambdaw a) \lhd_i t&:= \lambdaw a \lhd_{i'} t\upshift
        \end{cases}
$$
and use the shorthand $a\lhd b = a\lhd_0 b$. The substitution operator is left
associative i.e. $a \lhd_i b \lhd_k c$ is parsed as $(a \lhd_i b) \lhd_k c$.
\end{definition}


% Substitution swap lemma
% --------------------
\begin{theorem}
  It is possible to swap substitutions with the following equality
  $$
  \rulev
  {k \le n}
  {a \lhd_k b \lhd_n c = a \lhd_{n'} c \lhd (b \lhd_n c)}
  $$
  \begin{proof} Induction on the structure of $a$.
    \begin{enumerate}
    \item
      Goal $x_i \lhd_k b \lhd_n c = x_i \lhd_{n'} c \lhd (b \lhd_n c)$.
      \begin{enumerate}
      \item Case $i < k$: Trivial, because nothing is substituted.
      \item Case $i = k$:\\
        $\begin{array}{lll}
            x_k \lhd_k b \lhd_n c &=& b\upshift^k \lhd_n c \cr
            &=& x
          \end{array}
          $
      \item
      \end{enumerate}
    \item
    \item
    \end{enumerate}
  \end{proof}
\end{theorem}


% Beta reduction
% -----------
\begin{definition} \emph{Beta reduction} $\tobeta$ is a relation defined over
  unique lambda terms by the rules
  \begin{enumerate}
  \item $(\lambdaw a) b \tobeta a \lhd b$
  \item $\rulev{a\tobeta b}{a c \tobeta b c}$
  \item $\rulev{b\tobeta c}{a b \tobeta a c}$
  \item $\rulev{a \tobeta b}{\lambdaw a \tobeta \lambdaw b}$
  \end{enumerate}
\end{definition}


Test $\ulcorner a \urcorner $. Or $\ulcorner a\, \ulcorner a \urcorner \urcorner $.