\section{Computable Functions}
\label{sec:computable}

\subsection {Boolean Functions}

\paragraph{Boolean Values}

In lambda calculus there are no primitive data types. All lambda terms are
functions. If we want to represent boolean values in lambda calculus we have
to ask \emph{What can be done with a boolean value?}

Evidently boolean values can be used to decide between alternatives. We can
make the convention that the boolean value \emph{true} always decides for the
first alternative and the boolean value \emph{false} always decides for the
second alternative.

In the section \emph{Lambda Terms}~\ref{sec:lambda} we have already seen the
combinator kestrel $K$ which always returns the first argument of two
arguments and the kite $K_I$ which always returns the second argument of two
arguments. Therefore we use the definitions
$$
\begin{array}{lll}
  \true  & := & K \\
  \false & := & K_I
\end{array}$$


\begin{definition}
  A lambda term $t$ defines a \emph{boolean value} if it reduces to $\true$ or
  $\false$ in zero or more steps i.e. if
  $t \tobetastar
  \begin{cases} \true \\ \false
  \end{cases}$
  is valid.
\end{definition}



\paragraph{Boolean Functions}

\begin{definition}
  A lambda term $t$ represents an $n$-ary boolean function if given $n$
  arguments $b_1, b_2, \ldots b_n$ reduces to a boolean value i.e. if
  $$t b_1 b_2 \ldots b_n \tobetastar
  \begin{cases} \true \\ \false
  \end{cases}$$
  is valid.
\end{definition}



\paragraph{Negation}

Boolean negation must be a function taking one boolean argument and returning
a boolean value which is a two argument function representing a choice.

Remember the vireo which has the specification $V a b f \tobetaplus f a
b$. The term $V\, \false\, \true$ stores the two boolean values and waits for the
function to apply it to the two stored values. If we provide a boolean value
it selects $\false$ in case its value is $\true$ and $\true$ in case its value
is $\false$. This is exactly the specifaction of boolean negation. Therefore
we define
$$ (\lnot) := V\, \false\, \true.$$

Note that we use the logical symbol $\lnot$ to represent the lambda term for
boolean negation and we use the same symbol to denote negation on a logical
level. It should be clear from the context wheather the lambda term or the
logical symbol is meant.


\paragraph{Conjunction} The expression $a \land b$ where $a$ and $b$ represent
boolean values shall return $b$ in case that $a$ represents $\true$ and $a$
otherwise. Its nearly trivial to define a lambda expression which has exactly
this behaviour
$$ (\land) := \lambda a b. a b a.$$

\paragraph{Disjunction} The lambda term representing disjunction can be
defined as
$$ (\lor) := \lambda a b. a a b.$$
You can verify the validity of this definition by applying the definition to
all 4 possible cases of the truth table of disjunction.




\subsection{Composition of Decomposition of Pairs}

In order to represent pairs in lambda calculus we have to ask the question
\emph{What can be done with a pair?}. The most natural answer: \emph{Extract
  either the first or the second element}.

We have seen already the vireo $V$ which stores two values and waits for the
third argument to apply the third argument to the first two values. We have
the kestrel $K$ to select the first element and the kite $K_I$ to select the
second element. Therefore we can define the lambda terms
$$
\begin{array}{lll}
  \pair & := & V \\
  \fst   & := &\lambda p. p K \\
  \snd  & := &\lambda p. p K_I \\
\end{array}.
$$

Let us verify that the definitions are correct
$$
\begin{array}{llll}
  \fst\, (\pair\, a\, b)  &=& (\lambda p. p K) (V a b)  & \text{definitions} \\
  & \tobeta & (pK)[p:=V a b]      &\beta\text{-reduction}\\
  & = & V a b K & \text{substitution} \\
  & \tobetaplus & K a b  &\text{specification of } V \\
  & \tobetaplus & a & \text{specification of } K
\end{array}
$$

In the same manner the validity of $\snd\,(\pair\, a\, b) \tobetaplus b$ can
be verified.



\subsection{Numeric Functions}

Numbers are represented by lambda terms as iterations. The lambda term $c_n$
representing the natural number $n$ takes two arguments, a function $f$ and a
start value $a$ and iterates the function $f$ $n$ times beginning with the
start value $a$.

\begin{definition}
The $n$ iteration of the function $f$ on the start value $a$ is defined as
$$
f^n a :=
\begin{cases}
  f^0 a &:= a\\
  f^{n'} a & := f (f^n a)
\end{cases}
$$
where $n'$ denotes the successor of the natural number $n$.
\end{definition}


\paragraph{Church Numeral}

\begin{definition}
  The church numeral $c_n$ defined as
  $$ c_n := \lambda f a. f^n a$$
  is the lambda term representing the natural number $n$.
\end{definition}

\begin{definition}
  The lambda term $t$ represents a number if it reduces in zero or more steps
  to a church numeral i.e. if $$t \tobetastar c_n$$ is valid for some $n$.
\end{definition}

\begin{definition}
  The term $t$ represents an $n$-ary numeric function if given $n$ arguments
  $a_1, a_2, \ldots, a_n$ each one representing a number reduces in zero or
  more steps to a church numeral i.e. if $$t a_1 a_2 \ldots a_n \tobetastar
  c_m$$ is valid for some number $m$.
\end{definition}


\begin{definition}
  The term $t$ represents an $n$-ary numeric predicate if given $n$ arguments
  $a_1, a_2, \ldots, a_n$ each one representing a number reduces in zero or
  more steps to a boolean value i.e. if
  $$t a_1 a_2 \ldots a_n \tobetastar
  \begin{cases} \true \\ \false
  \end{cases}
  $$ is valid.
\end{definition}


\paragraph{Zero Tester}

A zero tester is a unary predicate on church numerals. It should return
$\true$ if the number is the church numeral $c_0$ and $\false$ for all other
church numerals.

Remember that church numerals are functions with a function argument $f$ and a
start value argument $a$ which iterate the function $n$ times starting at
$a$. Therefore $c_0 f a$ always returns $a$ regardless of the function $f$. So
our zero tester can have the shape $$\lambda x. x f\, \true$$ which does already
the right thing if applied to the argument $c_0$.

Now we need a lambda term for the iterated function $f$. The kestrel $K$
applied to two arguments always returns the first argument. If applied only to
one argument, it stores the argument and spits it out if it is applied to the
second argument. Therefore $K\,\false$ always returns $\false$ for any
argument. So we can use $K\,\false$ as the function argument $f$ in the above
template.

A valid definition of a zero tester is
$$ 0? := \lambda x. x (K\, \false) \, \true.$$

Proof:
$$
\begin{array}{lll}
  0?\, c_0
  &= &(\lambda x. x (K\, \false) \, \true) c_0\\
  &\tobeta & (x (K\, \false) \, \true)[x:=c_0] \\
  &= & c_0 (K\, \false) \, \true \\
  &\tobetaplus & \true \\
  \\
  0?\, c_{n'}
  &=& (\lambda x. x (K\, \false) \, \true) c_{n'}\\
  &\tobeta & (x (K\, \false) \, \true)[x:=c_{n'}] \\
  &= & c_{n'} (K\, \false) \, \true \\
  &=&  (K\,\false)  ( (K\,\false)^n\, \true) \\
  &\tobeta & \false
\end{array}
$$

\paragraph{Successor Function}

A lambda term representing the successor function takes a church numeral and
returns a church numeral representing the successor of its argument. The
signature of the successor function must be
$$ \lambda x f a. \cdots$$
where the first argument $x$ is the church numeral and the result must be a
church numeral i.e. a function taking two arguments.

We know that the term $x f a$ where $x$ represents a church numeral already
does $n$ iterations of $f$ on $n$. The successor function just has to do one
more iteration
$$
\succ := \lambda x f a. f (x f a).
$$

We prove the desired property $\succ\, c_n \tobetastar c_{n'}$ by
$$
\begin{array}{llll}
    \succ\, c_{n} &= & (\lambda x f a. f (x f a)) c_{n} & \text{definition} \\
     & \tobeta & \lambda f a. f (c_{n} f a) & \beta\text{-reduction} \\
     & = & \lambda f a. f (f^{n} a) & \text{definition of }c_{n} \\
     & = & \lambda f a. f^{n'} a) & \text{definition of }f^n a \\
     & = & c_{n'} & \text{definition of }c_{n'}
\end{array}
$$






\subsection{Primitive Recursion}

The addition of natural numbers is defined recursively as
$$
(+) :=
\begin{cases}
  0 + m &:= m \\
  n' + m &:= (n + m)'
\end{cases}
$$


However lambda calculus does not work on numbers, it works on church numerals
representing numbers. Wouldn't it be nice to define a lambda term $(+)$
representing to addition of two lambda terms representing numbers as

$$
(+) :=
\begin{cases}
  c_0 + m &:= m \\
  c_{n'} + m &:= \succ\, (n + m)
\end{cases}
$$

And indeed, this is possible. We just have to explain how the definition is
mapped into lambda calculus.

Note that, strictly spoken, the expression $a + b$ where $a$, $b$ and $(+)$
are lambda terms is not a lambda term according to our syntactic
definition. In order to be precise we have to define a lambda term
``$\text{plus}$'' which represents the addition function and write
``$\text{plus}\, a \, b$'' instead of the expression $a + b$. However the
latter is better readable and therefore we use it to denote the more
cumbersome expression ``$\text{plus}\, a \, b$''.

For convenience we use in the following the notation $\mathbf{x}$ to represent
$n$ arguments $x_1 x_2\ldots x_n$ and $g \mathbf{x}$ to represent the function
application $g x_1 x_2 \ldots x_n$ i.e.
$$
\begin{array}{lll}
  \mathbf{x} & := x_1 x_2 \ldots x_n\\
  g \mathbf{x} & := g x_1 x_2 \ldots x_n
\end{array}
.
$$

Suppose we have a lambda term $g$ representing an $n$-ary numeric function and
a lambda term $h$ representing an $n''$-ary numeric function. Then in order to
define an $n'$-ary function $f$ we can write
$$
f :=
\begin{cases}
  f c_0 \mathbf{x} &:= g \mathbf{x}\\
  f c_{n'} \mathbf{x} &:= h (f c_n \mathbf{x}) c_{n'} \mathbf{x}
\end{cases}
$$
and interpret this definition as an iteration in the following manner:
\begin{itemize}
  \item
    A pair of to church numerals $c_i$ and $c_j$ is used to represent the
    state of the iteration. The first numeral $c_i$ represent the number of
    iterations done and the second numeral represents the intermediary result.

  \item
    The iteration is started with $p_0 := \pair\, c_0 \, (g\mathbf{x})$.

  \item
    The step function is represented by the lambda term
    $$s := \lambda p. \pair\, i (h j i \mathbf{x})$$
    where $i:= \succ (\fst\, p)$ and $j := \snd\, p$.

    The step function extracts the iteration counter and the intermediate
    result from the pair $p$ and constructs a new pair by applying the
    successor function to the iteration counter and the function $h$ to the
    intermediate result and the remaining arguments.

  \item
    The function $f$ is defined by the lambda term
    $$ f := \lambda y \mathbf{x}. y \, s \, p_0$$
    where $y$ is a church numeral representing the first argument and
    $\mathbf{x}$ is the array of church numerals representing the remaining
    $n'$ arguments. Remember that the expression $y s p_0$ iterates the step
    function $s$ over its intial value $p_0$ as long as $y$ is a church numeral.
\end{itemize}


\subsection{Some Primitive Recursive Functions}

With the method of the last section we can define a lot of numeric functions.

\paragraph{Addition}
$$ (+) :=
\begin{cases}
  c_0 + m := m \\
  c_{n'} + m := \succ\, (c_n + m)
\end{cases}
$$

\paragraph{Multiplication}
$$ (\times) :=
\begin{cases}
  c_0 \times m := c_0 \\
  c_{n'} \times m := m + c_n \times m
\end{cases}
$$

\paragraph{Exponentiation}
$$ \bullet^\bullet :=
\begin{cases}
  a^{c_0}  := c_1 \\
  a^{c_{n'}} := a \times a^{c_n}
\end{cases}
$$

\paragraph{Factorial}
$$ (!) :=
\begin{cases}
  c_0!  := c_1 \\
  c_{n'}! := c_{n'} \times c_n!
\end{cases}
$$

\paragraph{Predecessor}
$$ \pred :=
\begin{cases}
  \pred\, c_0 := c_0 \\
  \pred\, c_{n'} := c_n
\end{cases}
$$


\paragraph{Difference}
$$ (-) := \lambda a b. b\, \pred\, a $$

The term $c_n - c_m$ applies the predecessor function $m$ times to $c_n$. The
result is the difference if $n > m$ or $c_0$ if $n \le m$.

\paragraph{Comparison}
$$ (\le) := \lambda a b.\, 0?\, (a - b) $$

\paragraph{Strict Comparison}
$$ (<) := \lambda a b.\, \succ\, a \le b $$


\paragraph{Numeric Equality}
$$ (\equiv) := \lambda a b.\, a \le b \land b \le a$$

The symbol $=$ is already reserved for exact equality or $\alpha$-equivalence
of lambda terms. Therefore the symbol $\equiv$ is used to denote the lambda
term which represents the equality function for church numerals.


\paragraph{Bounded Minimization} Let $g$ be a lambda term representing an
$n'$-ary predicate over church numerals. Then it makes sense to ask, if there
is a least number $y$ which makes $g y \mathbf{x}$ $\true$. The expression
$\mu^y g \mathbf{x}$ should return this least number or $y$ in case that there
is no number strictly below $y$ such that $g y \mathbf{x}$ is satisfied.

$$\mu :=
\begin{cases}
  \mu^{c_0} g \mathbf{x}    & :=  c_0 \\
  \mu^{c_{n'}} g \mathbf{x} & :=
  \big(\lambda z.\, (g z \mathbf{x}) z (\succ\, z)\big) (\mu^{c_n} g \mathbf{x})
\end{cases}
$$

Once $\mu^{c_n} g \mathbf{x}$ satisfies the predicate, the number remains
constant throughout the iteration. As long as $\mu^{c_n} g \mathbf{x}$ does
not yet satisfy the predicate, the successor is tried until the predicate is
satisfied or upper limit is reached.



\paragraph{Division}
$$ (\div) := \lambda a b.\, \mu^a\, (\lambda x. a < \succ\, x \times b)$$

This function only works correctly if the divisor $b$ is not zero. It computes
the least church numeral $x$ such that $\succ\,x \times b$ is greater than the
church numeral $a$. In that case the church numeral $x \times b$ is exactly
$a$ or leaves some remainder less than $b$.




\paragraph{Divides Exactly} The term $a\mid b$ shall return $\true$ if $a$
divides $b$ without remainder, otherwise it shall return $\false$. The
definition
$$ (\mid) :=
\lambda a b.\,
\neglow a\equiv c_0 \land
(b \div a)\times a \equiv b
$$
satisfies this requirement.




\paragraph{Prime Number Tester}
$$\Pr? :=
\lambda x.\, c_2 \le x \land
x \equiv \mu^x (\lambda z.\, c_2 \le z \land z\mid x)$$

The term $\mu^x (\lambda z.\, c_2 \le z \land z\mid x)$ computes the least
church numeral $z$ strictly below $x$ which is greater or equal $c_2$ and
divides $x$ exactly. If this number does not exist, the term computes $x$. In
that case $x$ is a prime number.



\paragraph{i-th Prime Number} We need a function $\Pr$ such that $\Pr c_0$
computes $c_2$, $\Pr c_1$ computes $c_3$, $\Pr c_2$ computes $c_5$ \ldots

If $z$ is a prime number then there is a prime number between $z$ and
$z!'$. We can use this fact to define $\Pr$ recursively.
$$
\Pr :=
\begin{cases}
  \Pr c_0    &:= c_2 \\
  \Pr c_{n'} &:=
  \big(\lambda z.\,
  \mu^{\succ\ z!}(\lambda y.\, \Pr y \land z < y)\big)
  (\Pr c_n)
\end{cases}
$$



\paragraph{Prime Exponent} If we have a church numeral $x$ we want to be able
to compute the exponent $e$ of the $i$th prime number such that $(\Pr i)^e$
divides $x$ exactly. The term $\Prexp i x$ shall compute the exponent.

The definition
$$ \Prexp :=
 \lambda i x.\, \mu^x (\lambda e.\, \neglow  (\Pr i)^{\succ\, e} \mid x)
$$
satisfies that requirement.


\paragraph{Encode Pairs of Church Numerals into a Church Numeral}

We can map a pair of natural numbers $n$ and $m$ into another natural number
by the formula
$$ 2^n (2m+1) -1.$$
It is not too difficult to see that both numbers $n$ and $m$ can be recovered
from a number $k$ to which the pair has been mapped i.e. that the mapping is
bijective. The number $n$ is the exponent of the prime factor of $2$ in
$z+1$. Then the number $m$ can be found in an obvious way from
$\frac{z+1}{2^n}$.

We have already defined all functions to compute the mapping $\sigma_2$ and its two
inverses $\sigma_{21}$ and $\sigma_{22}$ as lambda terms.
$$
\begin{array}{lll}
  \sigma_2 &:= &\lambda a b.\,  {c_2}^a \times (c_2 \times b + c_1) - c_1
  \\
  \sigma_{21} &:=& \lambda x.\, \Prexp c_0 (\succ\, x)
  \\
  \sigma_{22} &:=& \lambda x.\,(\pred\, (\succ\, x\div {c_2}^{\sigma_{21} x}))
                   \div c_2
\end{array}
$$

Note that this encoding of pairs of church numerals is completely different
from the lambda term $\pair$ and its inverses $\fst$ and $\snd$. The latter
functions encode pairs of arbitrary lambda terms and extract the first and the
second component of the pair while the functions $\sigma_2$, $\sigma_{21}$ and
$\sigma_{22}$ perform an arithmetic encoding of pairs of numbers and extract
the first and the second number of the pair of numbers.




\subsection{General Recursion}

\paragraph{The Problem} Suppose we have a lambda term $g$ representing an
$n'$-ary predicate and we know that for all arrays of church numerals
$\mathbf{x}$ there exists a church numeral $y$ such that $g y \mathbf{x}$ is
satisfied.

If we had the ability to program a loop in lambda calculus we could start an
unbounded iteration at the church numeral $c_0$ and stepwise increase the
number by one until the predicate $g$ is satisfied.

Unfortunately we just know that a number exists, but we are not able to
specify any bound in order to use the bounded $\mu$ operator as done in the
previous chapter.

\paragraph{Step Function} In order to program some unbounded search in lambda
calculus we need a function which performs one step in this search, i.e.
\begin{itemize}
\item Check if the boolean expression $g y \mathbf{x}$ is satisfied.
\item If yes, return $y$.
\item If no, apply a next step on $\succ\, y$.
\end{itemize}

The last point would be a recursive call. Unfortunately we have no direct
means to perform a recursive call in lambda calculus. Therefore we define the
step function $s$ in a way that it receives the function $f$ which does the
next step as an argument
$$ s := \lambda g \mathbf{x} f y.\, (gy\mathbf{x}) y \big(f (\succ\, y)\big).$$


\paragraph{Turing Combinator} Now we need a method to perform the unbounded
search. Remember the specification of the turing combinator
$$ U a f \tobetaplus f (a a f).$$

If we use instead of the term $a$ the turing combinator $U$ and add an
additional argument $y$ we get the
potential infinite loop
$$ U U f y\tobetaplus f (U U f) y \tobetaplus f (f (U U f)) y \tobetaplus
\cdots $$

Now let's see what happens if we use $s g \mathbf{x}$ for $f$  and $c_m$ for $y$:
$$
\begin{array}{lll}
  U U f c_m  &\tobetaplus & s g\mathbf{x} (U U f) c_m\\
  &\tobetaplus&  (g c_m \mathbf{x}) c_m \big(U U f (\succ\, c_m)\big)
\end{array}.
$$

The term returns $c_m$ if $g c_m \mathbf{x}$ is satisfied. Otherwise it starts
the next iteration on $\succ\, c_m$.


\paragraph{Unbounded Minimization} Having all this, we can define the unbouded
$\mu $ operator which receives an $n'$-ary predicate $g$, an $n$ array of
church numerals $\mathbf{x}$ such that the $\mu $ operator returns the least
church numeral $y$ which satisfies the predicate $g y \mathbf{x}$ if such a
term exists
$$ \mu := \lambda g \mathbf{x}.\, U U (s g \mathbf{x}) c_0$$
where $s$ is the above defined step function.
