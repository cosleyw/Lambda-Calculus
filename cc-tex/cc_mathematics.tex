\section{Basic Mathematics}
\label{sec:BasicMathematics}

\begin{comment}
    - Logic:
        - Rule notation
        - Universal quantification

    - Inductive Sets
\end{comment}



\subsection{Sets and Relations}

It is assumed that the reader is familiar with the following concepts:

\begin{enumerate}

\item Basic set notation: Elements of a set $a \in A$, Subsets $A \subseteq B$
defined as $a \in A$ implies $a \in B$.

\item Endorelations: An (endo-) relation $r$ over the set $A$ is the set of pairs
    $(a,b)$ (i.e. $r \subseteq A \times A$) where $a$ and $b$ are drawn from the
    set $A$. We write $(a,b) \in r$ more pictorially as $a \torel r b$.

\item $n$-ary Relations: An endorelation is a binary relation where the domain
    and the codomain are the same. An $n$-ary relation over the domains $A_1,
        A_2, \ldots, A_n$ is a subset from the cartesian product $A_1 \times A_2
        \times \ldots \times A_n$. In this paper we use a ternary relation as
        the typing relation.

\item An intuitive understanding of the reflexive transitive closure $r^*$ of a
relation $r$, the reflexivity, symmetry and transitivity of a relation and
similar concepts.

\item Logic: The logical connectives of conjunction $a \land b$, disjunction $a
\lor b$, implication $a \imp b$ and existential $\exists x. p(x)$ and universal
quantication $\forall x.p(x)$. We use the symbol $\perp$ for falsity i.e. a
logical statement which can never be proved (a contradiction).

\item An intuitive understanding of mathematical functions from the set $A$ to
the set $B$. I.e. $A \to B$ is the set of functions mapping each element of the
set $A$ to a unique element of the set $B$.
\end{enumerate}





\subsection{Logic}

We often have to state that some premises $p_1, p_2, \ldots p_n$ have a certain
conclusion $c$
$$
p_1 \land p_2 \land \ldots \land p_n \imp c
$$

In many cases this notation with logical connectives is clumsy and not very
readable. We use the rule notation
$$
\ruleh{p_1 \\ p_2 \\ \ldots \\ p_n}{c}
$$
to express the same fact.


If some variables in logical statements are not quantified, then universal
quantification is assumed.
$$
\begin{array}{ccc}

    \ruleh{a \in A}{a \in B}
    &
    \text{means}
    &
    \allbracket a {\ruleh {a \in A}{a \in B}}

\end{array}
$$

Assume that a logical statement has more than one variable. The universal
quantification can be expressed at the highest level or pushed down to the first
appearance of the variable. Moreover logical conjunction is commutative.
Therefore the following interpretations are equivalent.

$$
\begin{array}{ccc}
    \ruleh{a \in A \\ b \in A}{a \torel r b}
    &
    \text{means}
    &
    \begin{array}{l}
        \allbracket{ab}{\ruleh{a \in A \\ b \in A}{a \torel r b}}
        \\
        \allbracket a
        {
            \ruleh
            {a \in A}
            {\allbracket b {\ruleh{b \in A}{a \torel r b}}}
        }
        \\
        \allbracket b
        {
            \ruleh
            {b \in A}
            {\allbracket a {\ruleh{a \in A}{a \torel r b}}}
        }
    \end{array}
\end{array}
$$

All three settings of the universal quantifiers are logically equivalent.




\subsection{Inductively Defined Sets}

Sets can be defined inductively by rules. E.g. we can define the set of even
numbers by the rules

\begin{enumerate}
\item $0$ is an even number

\item If $n$ is an even number, then $n+2$ is an even number as well.
\end{enumerate}

We write this more formally:

\begin{quote}

\emph{The set of even numbers $E$ is defined by the rules}:
    \begin{enumerate}
    \item $$ 0 \in E$$

    \item
        $$
        \ruleh{n \in E}{n+2 \in E}
        $$
    \end{enumerate}
\end{quote}

We say that $E$ is the smallest set which satisfies the above rules.

Whenever there is an element $m \in E$ it can be in the set only if it is in the
set because of one of the two rules which define the set. $0$ is in $E$ because
of the first rule. $2$ is in $E$ because of second rule where $n = 0$ and $0$ is
in the set. $4$ is in the set because of the second rule where $n = 2$ and $2$
is in the set.

A number $m \in E$ can be arbitrarily big, but it can be in the set only because
of finitely many applications of the rules which define the set.




\subsection{Induction Proofs}

With inductively defined sets we can do induction proofs. Let's say that we want
to prove that all numbers $n$ in $E$ satisfy a certain property $p(n)$ i.e. we
want to prove
$$
\ruleh{n \in E}{p(n)}
$$

We prove this statement by induction on $n \in E$. Similar to induction proofs
on natural numbers we have to prove

\begin{enumerate}
\item $0$ satisfies $p$, i.e. $p(0)$

\item Every $n$ that satisfies $p$ (i.e. $p(n)$) must imply that $n+2$ satisfies
$p$ as well (i.e. $p(n+2)$).

\end{enumerate}

Because the second rule is recursive we get an induction hypothesis.

We write all induction proofs on an inductively defined set in the following
manner:

\begin{quote}
    Theorem: \emph{All even numbers $n \in E$ satisfy $p(n)$}.
    $$
    \ruleh{n \in E}{p(n)}
    $$

    Proof: By induction on $n \in E$.
    \begin{enumerate}
    \item
        $$
        \begin{array}{l|l}
        0 \in E
        & p(0)
        \end{array}
        $$
        $p(0)$ is valid because $\ldots$

    \item
        $$
        \begin{array}{l|l}
            n \in E
            &
            p(n)
            \\
            \hline
            n+2 \in E
            &
            p(n+2)
        \end{array}
        $$
        In order to prove $p(n+2)$ in the lower right corner we assume $n \in
        E$, $p(n)$, $n+2 \in E$. $p(n+2)$ is valid because $\ldots$
    \end{enumerate}
\end{quote}

For each rule in the inductive definition of the set we write a matrix. The left
part of the matrix is just the rule. The goal is always in the lower right corner
of the matrix. For each recursive premise (i.e. that some other elements are
already in the set) we get an induction hypothesis.

In order to prove the goal in the lower right corner we can assume all other
statements in the matrix.

This schema helps to layout all premises and all induction hypotheses precisely.

As an example we prove that for all even numbers $n \in E$ there exists a number
$m$ such that $n = 2m$ using the above schema.

\begin{quote}
    Theorem: \emph{For all even numbers $n \in E$ there exists a number $m$ with
    $n = 2m$}.
    $$
    \ruleh{n \in E}{\exists m. n = 2m}
    $$

    Proof: By induction on $n \in E$.
    \begin{enumerate}
    \item
        $$
        \begin{array}{l|l}
        0 \in E
        & \exists m. 0 = 2m
        \end{array}
        $$
        $m = 0$ satisfies the goal.

    \item
        $$
        \begin{array}{l|l}
            n \in E
            &
            \exists i. n = 2i
            \\
            \hline
            n+2 \in E
            &
            \exists m. n+2 = 2m
        \end{array}
        $$
        By the induction hypothesis we get a number $i$ which satisfies $n =
        2i$. Therefore $n+2 = 2i + 2 = 2(i+1)$. The number $m = i+1$ satisfies
        the goal in the lower right corner.
    \end{enumerate}
\end{quote}


Note that the name of the bound variable in existential and universal
quantification is arbitrary. It is good practice to choose names which do not
interfere. In the case for the recursive rule we have chosen the different names
$i$ and $m$ in the existential quantification. This makes the proof more
readable for humans.







\subsection{Inductively Defined Relations}

An (endo-) relation $r$ over the set $A$ is just a subset of the cartesian
product $A \times A$. I.e. a relation is just a set which can be defined
inductively. Since inductively defined relations are used excessively in the
paper we explain inductive definitions and induction proofs on relations as
well.

Note that the following is not restricted to binary endorelations over a carrier
set $A$. It is possibly to define $n$-ary relations over different carrier sets
as subset of the cartesian product $A_1 \times A_2 \times \ldots \times A_n$ as
well. E.g. the typing relation in the calculus of constructions (see
section~\ref{sec:Typing}) is a ternary relation.

As an example we define the reflexive transitive closure of a relation $r$
inductively.

\begin{quote}
    \emph{The reflexive transitive closure $r^*$ of the relation $r$ is defined
    by the rules}
    \begin{enumerate}
    \item
        $$ a \torel {r^*} a $$

    \item
        $$
        \rulev{
            a \torel {r^*} b
            \\
            b \torel r c
        }
        {
            a \torel {r^*} c
        }
        $$
    \end{enumerate}
\end{quote}


It should be intuitively clear that $r^*$ is reflexive and transitive. The first
rule guarantees reflexivity and the second rule guarantees transitivity. While
reflexivity is expressed explicitely in the first rule, transitivity is expressed
only indirectly in the second rule therefore we need a proof of transitivity.

\begin{quote}
    \def\rstar{{r^*}}

    Theorem: \emph{The reflexive transitive closure $r^*$ of the relation $r$ is
    transitive}
    $$
    \rulev{
        a \torel\rstar b
        \\
        b \torel\rstar c
    }
    {
        a \torel\rstar c
    }
    $$

    Proof: Assume $a \torel\rstar b$ and do induction on $b \torel\rstar c$.
    \begin{enumerate}
    \item
        $$
        \begin{array}{l|l}
            b \torel\rstar b
            &
            a \torel\rstar b
        \end{array}
        $$
    The goal $a \torel\rstar b$ is trivial because it is an assumption.

    \item
        $$
        \begin{array}{l|l}
            b \torel\rstar c_0
            &
            a \torel\rstar c_0
            \\
            c_0 \torel r c
            \\
            \hline
            b \torel\rstar c
            &
            a \torel\rstar c
        \end{array}
        $$
        By the induction hypothesis we have $a \torel\rstar c_0$. Together with
        the second premise $c_0 \torel r c$ and the second rule in the
        definition of $\rstar$ we get the goal in the lower right corner.
    \end{enumerate}
\end{quote}





\subsection{Term Grammar}

It is possible to define inductive sets by a grammar. E.g. we can define the set
of natural numbers $\Nat$ by the grammar


$$
\begin{array}{lllll}
    n
    &::= & 0 & \text{the number \emph{zero}}
    \\
    &\mid & n' & \text{the successor of $n$}
\end{array}
$$
where $n$ ranges over natural numbers and $0$ is a constant.

This grammar defines the natural numbers $\set{0, 0', 0'', \ldots}$ as strings
over the alphabet $\set{0, '}$.

A definition with a grammar is just a special form of an inductively defined
set.

\begin{quote}
    \emph{The set of natural numbers $\Nat$ is defined by the rules}
    \begin{enumerate}
    \item $$ 0 \in \Nat $$

    \item
        $$
        \ruleh{
            n \in \Nat
        }
        {
            n' \in \Nat
        }
        $$
    \end{enumerate}
\end{quote}


If we want to prove that all natural numbers $n$ satisfy a certain property $p$
we have to prove that $0$ satisfies the property (i.e. $p(0)$) and if $n$
satisfies the property then $n'$ has to satisfy the property as well i.e. $p(n)
\imp p(n')$.

This is the classical induction principle on natural numbers. Note that this is
just a special case of the more general induction proof scheme for inductively
defined sets.

An induction proof over terms defined by a term grammar has the following form:

\begin{quote}
    Theorem: \emph{All natural numbers $n$ have the property $p(n)$}.

    Proof: By induction on the structure of $n$:
    \begin{enumerate}

    \item $p(0)$: The goal $p(0)$ is valid because $\ldots$

    \item $p(n')$: By the induction hypothesis $p(n)$ is valid. This implies
    $p(n')$ because $\ldots$

    \end{enumerate}
\end{quote}

Note that such a proof corresponds one to one to a proof on inductive sets shown
above.

Furthermore note that a grammar rule in the term grammar might be recursive in
more than one subterm. In that case we get more than one induction hypothesis.




\subsection{Recursive Functions}

In the previous section we have shown that for each definition of a set with a
term grammar there is a corresponding inductive definition. However with terms
defined via a term grammar we can define recursive functions.

We use the natural numbers defined via a term grammar and define the recursive
function $p(n,m)$ which computes the sum of the numbers $n$ and $m$.
$$
\recfun
{p(n,m)}
{
    p(n,0)  &:=& n
    \\
    p(n,m') &:=& p(n,m)'
}
$$

The term grammar tells us that we can construct a natural number by using the
number $0$ and apply zero or more times the successor function $'$ to it. The
recursive function deconstructs the string generated by the term grammar. Since
every term has a finite length the recursion is guaranteed to terminate.
