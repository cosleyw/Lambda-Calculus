\section{Undecidability}

In the section \emph{Computable Functions}~\ref{sec:computable} we have
defined a class of numeric functions and predicates which can be computed in
lambda calculus. We call this class of functions/predicates lambda-computable.

Now the question arises: \emph{Are there functions or predicates which are not
  lambda computable}? The answer to this question is yes. This section of the
paper proves that there are undecidable predicates.

In order to prove the existence of undecidable predicates we use sets of
lambda terms $A$ which are closed and nontrivial. We make this precise by the
following definitions.

\begin{definition}
  A set of lambda terms $A$ is \emph{closed} if it contains with each lambda term $t$
  also all lambda terms which are $\alpha$- and $\beta$-equivalent to $t$.
\end{definition}


\begin{definition}
  A set of lambda terms $A$ is \emph{nontrivial} if it is neither empty nor
  does it contain all possible lambda terms.
\end{definition}

\paragraph{Gödel Numbering}
Since computable lambda function operate on church numerals we have to
transform a lambda term $t$ into a church numeral $\ulcorner t \urcorner$
which is a description of the lambda term $t$ in a way that having the church
numeral the corresponding term can be reconstructed. We define $\ulcorner t
\urcorner$ by
$$
\goedel{} :=
\begin{cases}
  \goedel{x_i} &:= \sigma_2 c_0 c_i \\
  \goedel{a b} &:= \sigma_2 c_1 (\sigma_2 \goedel{a} \goedel{b}) \\
  \goedel{\lambda x_i. a}
    &:=  \sigma_2 c_2 (\sigma_2 \goedel{x_i} \goedel{a})
\end{cases}
$$

Note that $\goedel{}$ is not a lambda term. It is easy, but very tedious, to
construct by hand for every lambda term $t$ the corresponding lambda term
$\goedel{t}$. However it is not possible to define a lambda term to do this
compilation because the lambda calculus cannot do pattern match on lambda
terms i.e. it cannot case split on the way the lambda term $t$ has been
constructed. Later we define a lambda term which can compute $\goedel{c_n}$
for church numerals $c_n$.

\begin{definition}
  A set $A$ of lambda terms is \emph{decidable} if there is a lambda term
  $p_A$ representing a unary predicate on church numerals such that $p_A
  \goedel{t}$ returns $\true$ if $t \in A$ and $\false$ if $t \notin A$.
\end{definition}

\paragraph{Self Application} Since lambda calculus does not have any
restrictions on functions and arguments (they only have to be valid lambda
terms) we can apply any lambda term $t$ to its description $\goedel{t}$
i.e. the term
$$ t \goedel{t}$$
is a legal lambda term.

Therefore we can define for all sets of lambda terms $A$ a set $B_A$ by
$$ B_A := \{b \mid b \goedel{b} \in A\}.$$

We assume that there is a lambda term $\self$ which satifies the specification
$$ \self\, \goedel{t} \tobetaplus \goedel{t \goedel{t}}.$$
A concrete definition of the lambda term $\self $ will be given later.

We can use the lambda term $\self $ to see that for every decidable set of
lambda terms $A$ the set $B_A$ is decidable as well. Proof: The term
$$\lambda x.\, p_A (\self\, x)$$
is a unary predicate which given a description $\goedel{b}$ of a term $b$
decides wheather $b$ is an element of $B_A$.

\paragraph{Basic Undecidability Theorem}

\begin{theorem}
  Every closed nontrivial set of lambda term $A$ is undecidable.
  \begin{proof}
    Assume that $A$ is decidable and $p_A$ is a lambda term which decides $A$.
    \begin{enumerate}

    \item
      There are the terms $m_0 \in A$ and $m_1 \notin A$, because $A$ is
      nontrivial.

    \item
      We define the lambda term $g$ by
      $$ g := \lambda x. p_A (\self\, x) m_1 m_0$$
      which has the property
      $$ g \goedel{b} \tobetaplus
      \begin{cases}
        m_1 & \text{if } b \in B_A \\
        m_0 & \text{if } b \notin B_A \\
      \end{cases}.
      $$

    \item
      Assuming $g \in B_A$ leads to a contradition
      $$
      \begin{array}{llll}
        g \in B
        & \imp & g \goedel{g} \tobetaplus m_1 & \text{definition of } g\\
        & \imp & g \goedel{g} \notin A  & A \text{ is closed} \\
        & \imp & g \notin B_A & \text{defintion of }B_A
      \end{array}
      $$

    \item
      Assuming $g \notin B_A$ leads to a contradition
      $$
      \begin{array}{llll}
        g \notin B
        & \imp & g \goedel{g} \tobetaplus m_0 & \text{definition of } g\\
        & \imp & g \goedel{g} \in A  & A \text{ is closed} \\
        & \imp & g \in B_A & \text{definition of }B_A
      \end{array}
      $$

    \item
      Therefore the assumption that $A$ is decidable cannot be valid.
    \end{enumerate}
  \end{proof}
\end{theorem}


\paragraph{Undecidability of Beta Equivalence}

\begin{theorem}
  Beta equivalence is undecidable.
  \begin{proof}
    Assume that beta equivalence is decidable.
    \begin{enumerate}
    \item
      Then there is some binary predicate $p$ such that for two lambda
      terms $a$ and $b$ the term $p \goedel{a} \goedel{b}$ returns $\true$ if
      they are beta equivalent and $\false$ if they are not equivalent.
    \item
      Let $A$ be the set of lambda terms which contains $a$ and all $\alpha$-
      and $\beta$-equivalent terms. Then by assumption the term $p \goedel{a}$
      is a decider for the set $A$.
    \item
      The set $A$ is nontrivial for the following reason: If $a$ is
      normalizing than $A$ can contain only normalizing terms. Therefore
      e.g. the term $M M$ where $M$ is the mockingbird combinator which is not
      normalizing is not in the set. If $a$ is not normalizing then $A$ cannot
      contain any term in normal form.
    \item
      The set $A$ is closed and nontrivial, therefore it cannot be decidable
      which contradicts the assumption that there is a decider for
      $\beta$-equivalence.
    \end{enumerate}
  \end{proof}
\end{theorem}



\paragraph{Undecidability of the Halting Problem}

Like the halting problem for Turing machines there is a halting problem for
lambda calculus. The halting problem is solvable if there is a lambda term
which determines if another lambda term is normalizing.

\begin{theorem}
  It is undecidable whether a lambda term $a$ is normalizing.
  \begin{proof}
    Assume that there is a lambda term $p$ such that $p \goedel{a}$ returns
    $\true$ if $a$ is normalizing and $\false$ if $a$ is not normalizing.
    \begin{enumerate}
    \item
      Let $A$ be the set which contains all lambda terms in normal form and
      all $\alpha$- and $\beta$-equivalent terms. By definition $A$ is closed
      and it is nontrivial (it contains all variables and it does not contain
      $M M$.)
    \item
      The lambda term $p$ would be a decider for $A$, because a term $a$ must
      be in this set if it is normalizing.
    \item
      The set $A$ however being closed and nontrivial cannot be decidable
      which contradicts the assumption that being normalizing is decidable.
    \end{enumerate}
  \end{proof}
\end{theorem}




\paragraph{Implementation of $\self $}

In the proof of the main undecidability theorem we used a lambda term $\self$
with the specification
$$ \self\, \goedel{a} \tobetaplus \goedel{a \goedel{a}}.$$

Now we give the still missing implementation of that term.
By definition of $\goedel{}$ the term we have the equality
$$\goedel{a \goedel{a}}
= \sigma_2 c_1 (\sigma_2 \goedel{a}
\goedel{\goedel{a}}).$$

The only unknown term on the right hand side of the equality is
$\goedel{\goedel{a}}$ which is the description of the church numeral
representing the lambda term $a$.

Any church numeral has the form
$$c_n = \lambda f x. f^n x.$$
Since the names of bound variables are irrelevant we use $x_0$ for $x$ and
$x_1$ for $f$.
By definition of $\goedel{}$ we get
$$
\begin{array}{lll}
  \goedel{x_0} &=& \sigma_2 c_0 c_0 \\
  \goedel{x_1} &=& \sigma_2 c_0 c_1
\end{array}.
$$
%
With the step function
$$ s := \lambda x.\, \sigma_2 c_2 (\sigma_2\, \goedel{x_1} x)$$
the term
$$ c_n s \goedel{x_0}$$
computes $\goedel{{x_1}^n x_0}$
and the function $f$ defined by
$$ f:= \lambda x. \sigma_2 c_2 (\sigma_2 \goedel{x_1} x)$$ computes
$\goedel{\lambda x_1.z}$ given $\goedel{z}$ as argument. I.e.
$$f (f (c_n s \goedel{x_0}))$$
computes the church numberal $\goedel{\lambda x_1 x_0. {x_1}^n x_0}$.
The term $\self$ can be defined by
$$ \self :=
\lambda x.\, \sigma_2 c_1 (\sigma_2 x (f (f (x s \goedel{x_0}))))
$$
